\documentclass[a4paper,10pt]{article}
% Importation de packages divers
   \NeedsTeXFormat{LaTeX2e} 
   \usepackage[usenames]{color}
   \usepackage[paper=a4paper,textwidth=160mm,twosideshift=0pt,hmargin=2.5cm, vmargin=1.5cm]{geometry}
   \usepackage{fancyhdr}
   \usepackage{lastpage}  
   \usepackage{wasysym}
   \usepackage{ marvosym }
   \usepackage{titlesec}
\titleformat{\section}[block]{\Large\bfseries\filcenter}{}{1em}{}
   \usepackage{float}
\restylefloat{table}
    % pour l'affichage du n� de la derni�re page.
   \usepackage[latin1]{inputenc}        % utilisation des caract�rres 8 bits en Unix (codage ISO 8859-1)
 %  \usepackage{floatflt,multirow}       % pour l'utilisation des figure ``noy�e'' dans le texte
   \usepackage[francais]{babel}         % Utilisation du fran�ais (nom des sections, c�sure, ponctuation, ...)
  % \usepackage{amsmath,amsthm,amssymb}  % Utilisation de certains packages de AMS (cf. belles �quations)
   \usepackage{fontenc}
   \usepackage[cyr]{aeguill}
   \usepackage{endnotes}                % Pour l'utilisation des notes en fin de documents
   \usepackage{verbatim}                % Pour l'insertion de fichier en mode verbatim
   \usepackage[pdftitle={Fichier realise a partir de base_lettre.tex},  % apparition ds les propri�t�s du doc
               pdfauthor={Fran�ois Rameau},
               pdfsubject={Qualification},
               pdfkeywords={Rameau},
	       colorlinks=true,
	       linkcolor=webdarkblue, 
	       filecolor=webblue, 
	       urlcolor=webdarkblue,
	       citecolor=webgreen]{hyperref}     % pour l'utilisation des liens http,...
   \usepackage{portland}		% pour l'utilisation de \portrait et de \landscape sur une page
\pagestyle{fancy}
%opening
\title{\LARGE{CANDIDATURE A LA QUALIFICATION
AUX FONCTIONS DE MAITRE DE CONFERENCES} \\
	\LARGE{Section CNU: 61\up{�me}} }
\author{\textbf{Guilaume \textsc{Lemaitre}}}
\date{08/12/2016}
\begin{document}

\maketitle
\thispagestyle{fancy}
  \lhead{Guillaume \textsc{Lemaitre}}
  \rhead{}
  \chead{Dossier de candidature}
  \cfoot{}
  \vspace*{3cm}
  \section*{Etat civil}
 \begin{table}[H]
 \center
 \begin{tabular}{l l} 
   Nom: & \textbf{Lemaitre} \\ \\ [-2ex]
   Pr�nom: & \textbf{Guillaume} \\ \\ [-2ex]
   Date et lieu de naissance: & \textbf{27 Avril 1988 � Decize (58)} \\ \\ [-2ex]
   Nationalit�: & \textbf{Fran�aise} \\ \\ [-2ex]
   Situation de famille: & \textbf{Mari\'e sans enfant} \\ \\ [-2ex]
   Coordonn�es professionnelles & \textbf{INRIA Saclay - PARIETAL} \\
   & \textbf{1 Rue Honor� d'Estienne d'Orves} \\
   & \textbf{91120 Palaiseau} \\
   & \phone \textbf{+33 7 61 10 47 82} \\
   & \Email \textbf{guillaume.lemaitre@inria.fr} \\ \\ [-2ex]
   Coordonn�es personnelles: & \textbf{86 rue Mar\'echal Foch} \\
   & 71200 \textbf{Le Creusot} \\
   & \Mobilefone \textbf{+33 7 61 10 47 82} \\
   & \Email \textbf{g.lemaitre58@gmail.com} 
\end{tabular}
 \end{table}
 
 %Vision omnidirectionnelle $\&$ hybride, Structure-From-Motion, Suivi-visuel, Auto-calibrage, Robotique mobile
 \section*{R�sum�}
\begin{table}[h]
\begin{tabular}{llll}
\multirow{}{}{Publications:}  & \textbf{Revues:}              & \textbf{5}   & 3 en 1\textsuperscript{er} auteur          \\
                                & \textbf{Conf�rences}          & \textbf{14}   & 7 en auteur correspondant       \\
                                & \textbf{Rapport technique}            & \textbf{1}   &                           \\ \\ [-2ex]
 
\multirow{}{}{Enseignements:} & \textbf{Base de donn\'ees}  & \textbf{40h} & DUT GEII  \\
                              & \textbf{Traitement du signal / Python} & \textbf{32h} & DUT MP \\
                                & \textbf{Software engineering}  & \textbf{24h} & Master ViBOT  \\
                                & \textbf{Introduction to image processing} & \textbf{56h} & Master ViBOT \\
                                & \textbf{Machine learning and pattern recognition} & \textbf{32h} & Master ViBOT \\
                                & \textbf{Medical Image Analysis} & \textbf{12h} & Master ViBOT  \\ 
\end{tabular}
\end{table}
Mots-cl�s: Medical Image Analysis, Pattern Recognition, Machine Learning, Computer-Aided Diagnosis, Prostate Cancer

            
\end{document}

